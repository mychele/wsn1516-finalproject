\section{Conclusions}
In this project we have implemented a system to transmit and receive a file over a noisy channel using network coding. Results show good performances, sufficient to allow a real utilization of the system (although clearly much more performing solutions exist).\\
We have compared two different algorithms for network coding: RF and LT, confirming that the crucial drawback of RF is its decoding time, which becomes unacceptable for large values of $N$ and $K$. We have also shown different trade-offs, the most important being the one between efficiency and goodput. \\
In this project we have also learned how to write a working systems, which requires, other than the main algorithms, also taking care of many minor details, but which may make the system fail if not accurately designed (e.g. retransmission protocols, timeouts). We have also learned to collect and arrange data in significant ways in order to meaningfully evaluate the performances of the system and highlight trade-offs.

