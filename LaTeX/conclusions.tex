\section{Conclusions}\label{sec:concl}
In this project we implemented a system to transmit and receive a file over a noisy channel using network coding. Results show good performance, sufficient to allow a real utilization of the system (although clearly much more performing solutions exist).\\
We have compared two different algorithms for network coding: RF and LT, confirming that the crucial drawback of RF is its decoding time, which becomes unacceptable for large values of $N$ and $K$. We have also shown different trade-offs, the most important being the one between efficiency and goodput, and how they depend on different parameters, mainly $K$ and the redundancy, but also timeouts.\\
In this project we have learned how to design and code a working system. This requires, other than the main algorithms, also taking care of many minor details, which however may make the system fail if not accurately designed (e.g. retransmission protocols, timeouts). In particular it also requires performance-oriented coding, which is not something trivial. For example, a good amount of effort was devolved into trying different solutions (e.g. vectors VS array, bitset VS array of booleans VS \texttt{mat\_GF2}) to find the most performing ones. We have also learned to automatically collect and arrange data in significant ways in order to meaningfully evaluate the performance of the system and highlight trade-offs.

