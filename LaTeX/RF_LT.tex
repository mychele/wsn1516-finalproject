\section{Implementation of RF and LT}
\subsection{RF}
Once the received has collected a suitable amount of packets, these packets are passed to the \texttt{decoding\_function}. This functions arranges the encoding vectors in a matrix and tries to invert it using Gau{\ss}-Jordan elimination. In order to get an efficient implementation of this algorithm in $\mathds{G}\mathds{F}(2)$ we used objects of class \texttt{mat\_GF2} which is implemented in library \texttt{NTL} ~\cite{NTL}. We then check if on left part of the matrix we have a complete identity matrix. If it's the case, then we extract from the right part of the matrix the first K rows, which give the inverse of the initial matrix. Using the inverse, we can then apply the inverse XOR transformation, on the dedoced packet payloads, and obtain the original uncoded payloads. In the other case, the function returns the number of missing rows to get K independent ones. The inversion algorithm is then re-applied with the original plus the additional packets (unfortunately, there is no way to exploit the partial results of the failed Gau{\ss}-Jordan elimination to reduce the computations for the successive trials: the complexity is the same of starting from scratch).
\subsection{LT}
For the implementation of LT, we have to represent the graph of message passing. This is done by using two adjacency lists. This choice have been made in order to minimize the computing time of the algorithm, even though it results quite expensive in terms of memory usage. Packets are then resolved according to the message passing algorithm. In this case, if the algorithm fails, there is no easy way to determine how many independent encoding vectors would still be needed in oder to complete the decoding. Because of this and the fact that theoretically, for K sufficiently large, this event would be very unlikely, encouraged us to opt for a simpler system, in which if the decoding fails, then the receiver simply asks for another group of N packets to the sender. 
