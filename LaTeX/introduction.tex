\begin{abstract}
Network coding techniques can be used to efficiently transmit data over wireless links and networks. In this project we will describe an implementation of a sender/receiver pair that by using network coding and ARQ mechanisms transmits a file over different kind of connections. We will show that the choice of different parameters outlines trade-offs in terms of different performance metrics, and that the system reaches high efficiency and goodput when used over a real connection from Padua to Lausanne.
\end{abstract}

\section{Introduction}
Network coding techniques emerged at the beginning of the century as an efficient way to transmit data over links with high packet error rate, and to avoid having different retransmissions for different multicast destinations. The basic principle is that the transmitted packets are encoded version of the original ones, which are combined together according to different rules, and piggyback also an encoding vector that specifies which packets were combined. The receiver is able to retrieve the original information by knowing the encoding vectors. If the transmission medium is not reliable and introduces packet losses, then new encoded packets created on the fly can be retransmitted, until the receiver is capable of decoding. This is why these codes are defined as rateless. The pioneer of network coding techniques is Michael Luby, which introduced as first the idea of Random Fountain (RF) \cite{rf} and then the Luby Transform (LT) codes \cite{lt}.

In this project we implemented a system composed of a sender and a receiver, which use either RF or LT network coding techniques, together with ARQ, in order to transmit a file from source to destination, adapting to different kind of transmission mediums (Wireless Local Area Network - WLAN, internet, localhost). The remainder of this report is organized as follows. Firstly, in Sec.~\ref{sec:impl} we will describe the implementation and the technical issues that emerged during our work. Secondly, Sec.~\ref{sec:results} will contain data and comments on a thorough experimental campaign we conducted in order to assess the performance of our implementation. Finally, some conclusions will be drawn in Sec.~\ref{sec:concl}.