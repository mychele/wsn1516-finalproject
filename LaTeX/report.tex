\documentclass[12pt, titlepage]{article}
%\usepackage[latin1]{inputenc}
\usepackage[T1]{fontenc}
\usepackage{cite, url,color} % Citation numbers being automatically sorted and properly "compressed/ranged". 
%\usepackage{pgfplots}
\usepackage{graphics,amsfonts}
\usepackage[pdftex]{graphicx}
\usepackage[cmex10]{amsmath}
\usepackage{ dsfont }
\usepackage{lipsum}
\usepackage{etex}
% Also, note that the amsmath package sets \interdisplaylinepenalty to 10000
% thus preventing page breaks from occurring within multiline equations. Use:
 \interdisplaylinepenalty=2500
% after loading amsmath to restore such page breaks as IEEEtran.cls normally does.
\usepackage[utf8]{inputenc}
%% Useful packages for creation of two-column and more complex figures
% Compact lists
\usepackage{enumitem}

\usepackage{array}
% http://www.ctan.org/tex-archive/macros/latex/required/tools/
\usepackage{mdwmath}
\usepackage{mdwtab}
%mdwtab.sty	-- A complete ground-up rewrite of LaTeX's `tabular' and  `array' environments.  Has lots of advantages over
%		   the standard version, and over the version in `array.sty'.
% *** SUBFIGURE PACKAGES ***
% deprecated \usepackage[tight,footnotesize]{subfigure}
%\usepackage{subfig}

\usepackage[top=2cm, bottom=2cm, right=1.6cm,left=1.6cm]{geometry}
\usepackage{indentfirst}
\usepackage{dsfont}
\usepackage{tikz}
\usepackage{pgfplots}
\pgfplotsset{compat=newest} 
\pgfplotsset{plot coordinates/math parser=false} 
\newlength\fheight
\newlength\fwidth

\usepackage{fancybox}
\usepackage[section]{placeins}
\usepackage{fancyvrb}
\usepackage{lmodern}
\usepackage{graphicx}
\usepackage{amsmath}
\usepackage{amssymb}
\usepackage{mathrsfs}
\usepackage{multirow}
\usepackage{wrapfig}
\usepackage{listings}
\usepackage{color}
\usepackage{geometry}
\usepackage{bbm}
%\usepackage{mnsymbol}
\usepackage{turnstile}
\usepackage{caption}
\usepackage{booktabs}
\usepackage{caption}
\usepackage{subcaption}
\usepackage{adjustbox}
\usepackage{rotating}
\usepackage{mwe}

\newcommand{\ap}[1]{\raisebox{0.6ex}{\scriptsize #1}} 
\newcommand{\ped}[1]{\raisebox{-0.6ex}{\scriptsize #1}}
\usepackage{lscape}
\usepackage{multicol}
% for tikz figures
\usetikzlibrary{automata,positioning,chains,shapes,arrows}
\usetikzlibrary{plotmarks}

\usepackage{times}
%\usepackage[active]{srcltx}
\graphicspath{{./figure/}}

\setlength\parindent{0pt}
\linespread{1}

\def\C#1{\mathcal{#1}}

\renewcommand{\phi}{\varphi}
% OPERATORS
\newcommand{\floor}[1]{{\left\lfloor #1\right\rfloor}}
\newcommand{\ceil}[1]{{\left\lceil #1\right\rceil}}
\newcommand{\E}[1]{\mathop{\rm E}\nolimits\left[#1\right]} % Expectation
\newcommand{\pr}[1]{\Pr\left[ #1 \right]}
\renewcommand{\P}[1]{\mathrm{P}\left[ #1 \right]}

\newcommand{\mbb}{\mathbb}
\newcommand{\mee}{\mathrm{e}}
\newcommand{\mrr}{\mathrm}
\newcommand{\m}{\mathrm{m}}
\newcommand{\mmin}[1]{{\min\left\{#1\right\}}}
\newcommand{\mmax}[1]{{\max\left\{#1\right\}}}
\newcommand{\id}[1]{\mathbf{1}\(#1\)} % Unit function  
\newcommand{\Heav}[1]{H\left(#1\right)} % Heaviside function
\newcommand{\eps}{\varepsilon}
\newcommand{\rect}[1]{\mathrm{rect}\!\left(#1\right)}
\newcommand{\sinc}[1]{\mathrm{sinc}\!\left(#1\right)}
\newcommand{\bin}[2]{{\left(\begin{array}{c}#1\\#2\end{array}\right)}}

\newcommand{\me}[2]{\left[ {#1} \right]_{(#2)}} % Submatrix \me{P}{i,j} produces [P]_{i,j} and denotes the element in the i-th row and j-th column of P
\newcommand{\vv}[1]{\left[ {#1} \right]} % Submatrix \vv{a,b,c} produces [a,b,c]

\newcommand{\Set}[1]{{\C #1}} 
\newcommand{\Setd}[2]{\Set{#1}=\left\{#2\right\}}  % Set definition: \Set{S}{0,1,2} produces S={01,2,} where S is in mathcal 

\newcommand{\mat}[1]{{\hbox{\textbf{#1}}}}
\newcommand{\ei}[1]{{\mat{e}_{#1}}}   % all zero vector except in the $#1$-th element which is one
\newcommand{\ind}[1]{\mathbf{\chi}\left\{#1\right\}} % Indicator function \ind{A}=1 if A is true, \ind{A}=0 otherwise. 

% FORMATTING
\newcommand{\ie}{i.e.,\,}
\newcommand{\eg}{e.g.,\,}
%\newcommand{\columnbreak}{\vfill\eject} % Column break

% REFERENCES
\newcommand{\Fig}[1]{Fig.~\ref{#1}}
\newcommand{\eq}[1]{(\ref{#1})}
\newcommand{\Tab}[1]{Tab.~\ref{#1}}
\newcommand{\Sec}[1]{Sec.~\ref{#1}}

\newenvironment{sistema}%
{\left\lbrace\begin{array}{@{}l@{}}}%
{\end{array}\right.}


\usepackage{layout}
\usepackage{changepage}

	
\definecolor{cerulean}{cmyk}{0.94,0.11,0,0}
\definecolor{processblue}{cmyk}{0.96,0,0,0}
\definecolor{Eored}{rgb}{.647 ,.129 ,.149}
\definecolor{Eogreen}{rgb}{0 ,0.53 ,0}
\definecolor{Eoblack}{rgb}{0 ,0 ,0}

\interfootnotelinepenalty=10000

\linespread{1.5} 

\begin{document}
\title{WSN - Final report on LT codes implementative project}
\author{Stefano Olivotto - Michele Polese}
\maketitle

% tikz styles
\tikzstyle{startstop} = [rectangle, rounded corners, minimum width=3cm, minimum height=1cm,text centered, draw=black]
\tikzstyle{io} = [trapezium, trapezium left angle=70, trapezium right angle=110, minimum width=3cm, minimum height=1cm, text centered, draw=black]
\tikzstyle{process} = [rectangle, minimum width=3cm, minimum height=1cm, text centered, draw=black]
\tikzstyle{decision} = [ellipse, minimum width=2cm, minimum height=1cm, text centered, draw=black]
\tikzstyle{arrow} = [thick,->,>=stealth]
\tikzstyle{darrow} = [thick,->,>=stealth,dashed]


\begin{abstract}
Network coding techniques can be used to efficiently transmit data over wireless links and networks. In this project we will describe an implementation of a sender/receiver pair that by using network coding and ARQ mechanisms transmits a file over different kind of connections. We will show that the choice of different parameters outlines trade-offs in terms of different performance metrics, and that the system reaches high efficiency and goodput when used over a real connection from Padua \SO{Padova (we always use Padova)} to Lausanne.
\end{abstract}

\section{Introduction}
Network coding techniques emerged at the beginning of the century as an efficient way to transmit data over links with high packet error rate, and to avoid having different retransmissions for different multicast destinations. The basic principle is that the transmitted packets are encoded version of the original ones, which are combined together according to different rules, and piggyback also \SO{and they also piggyback} an encoding vector that specifies which packets were combined. The receiver is able to retrieve the original information by knowing the encoding vectors. If the transmission medium is not reliable and introduces packet losses, then new encoded packets created on the fly can be retransmitted, until the receiver is capable of decoding. This is why these codes are defined as rateless. The pioneer of network coding techniques is Michael Luby, which introduced as first the idea of Random Fountain (RF) \cite{rf} and then the Luby Transform (LT) codes \cite{lt}.

In this project we implemented a system composed of a sender and a receiver, which use either RF or LT network coding techniques, together with ARQ, in order to transmit a file from source to destination, adapting to different kind of transmission mediums (Wireless Local Area Network - WLAN, internet, localhost). The remainder of this report is organized as follows. Firstly, in Sec.~\ref{sec:impl} we will describe the implementation and the technical issues that emerged during our work. Secondly, Sec.~\ref{sec:results} will contain data and comments on a thorough experimental campaign we conducted in order to assess the performance of our implementation. Finally, some conclusions will be drawn in Sec.~\ref{sec:concl}.
\section{Implementation of sender and receiver}
Sender and receiver are implemented as two standalone classes, which can be launched by a command line interface with some options. They create all the objects needed, then open some sockets and perform transmission and reception of a file using network coding techniques. We implemented two different versions, one which relies on a Random Fountain coding, and the second that uses LT codes, which is much more difficult to implement but offers better performances.

In order to easily handle encoded packets, we defined the class \texttt{NCpacket}. It has a single private variable, a \texttt{NCpacketContainer} struct that stores an header as a 32 bit integer, a sequence number (\texttt{blockID}) as an 8 bit char, and the payload (i.e. the encoded data) as char array of fixed size. It has constructors that accept either these three parameters or the \texttt{blockID} and the whole chunk of uncoded data, and performs the encoding inside the constructor. 
The header represents a seed which is given to a random number generator in order to create the same encoding vector at sender and receiver side. 

The RF implementation relies on C++ \texttt{rand()} to generate encoding vectors, and \texttt{NCpacket} objects are directly created in sender and receiver main methods.
The LT implementation, instead, uses a factory paradigm to generate \texttt{NCpacket} objects, i.e. it does not directly call the object constructor but creates an helper, \texttt{NCpacketHelper}, which is initialized when the main method of sender (and receiver) is called. This allows to generate only once the Robust Soliton Distribution needed to perform coding and the C++ objects of the \texttt{random} class, which allow a more robust approach for the encoding vector generation. This class has a method that from the seed generates a vector (of variable size) with the position of ones in the encoding vector (which is much more efficient than handling the whole encoding vector, with few ones and thousands of zeros).

The receiver uses objects from a \texttt{TimeCounter} class that performs estimation of time intervals, using the approach inspired to~\cite{rfc6298}. The time intervals of interest are time between the reception of two packets, in order to perform packet gap detection, and the RTT, in order to estimate whether an ACK sent to the sender was received or not. RFC~\cite{rfc6298} proposes a method to estimate RTT for TCP connection, based on some filtering of RTT measurements. However, the order of magnitude of the quantities of interest is much smaller than the minimum value that is returned by an estimator working with~\cite{rfc6298} rules. Therefore some changes have to be made. Let $s_{est}$ be the smoothed estimate of the quantity of interest, $s_{var}$ an estimate of the variance, $s$ a new measurement. Before any measurement is taken, $s_{est}$ is initialized at 50 ms and $s_{var} = s_{est}/2$. Then, once a new value $s$ is available, these two variables are updated as follows
\begin{equation}
	s_{var} = (1-\beta) s_{var}  + \beta |s_{est} - s|
\end{equation}
\begin{equation}
	s_{est} = (1 - \alpha) s_{est} + \alpha s
\end{equation}
where $\alpha = 1/8$, $\beta = 1/4$ as suggested in~\cite{rfc6298}.

The value returned by the \texttt{TimeCounter} object is finally
\begin{equation}
	\max\left\{1, s_{est} + K\times s_{var}  \right\} \mbox{ ms}
\end{equation}
where a granularity of minimum 1 ms is set (instead of 1 s as in~\cite{rfc6298} and $K = 4$. 

Finally, sender and receiver use some static functions which are place in a common \texttt{utils} class. 

Let's now discuss the implementation of sender and receiver. The retransmission policy is a stop and wait per block, i.e. until a block is not successfully received packets for that block are sent. Flow diagrams for sender and receiver are in Fig.~\ref{fig:sender} and~\ref{fig:receiver} respectively. 

\begin{sidewaysfigure}
\captionsetup{justification=centering,margin=3cm}

\begin{subfigure}{0.5\hsize}\centering
    \begin{tikzpicture}[node distance = 2cm, scale=0.8, every node/.style={scale=0.8}]
	\node (start) [startstop] {Init};
	\node (setup) [process, below of=start, align=center] {Open socket, initialize \\ useful variable, open input file};
	\node (filesetup) [process, below of=setup, align=center, yshift=-1cm] {Compute file size,\\ number of blocks $B$, \\ set number of tx block $b = 0$};
	\node (while) [decision, below of=filesetup, yshift=-1cm] {$b < B$?};
	\node (stop) [startstop, below left of=while, yshift=-1cm, xshift=-2cm] {Stop};
	\node (read) [process, below right of=while, yshift=-1cm, xshift=1.2cm, align=center] {Read data, set \\ packets needed $P = N$};
	\node (send) [process, below of=read] {Encode and send $P$ packets};
	\node (waitack) [process, left of=send, xshift=-4cm] {Wait for ACK};
	\node (ack) [decision, below of=send, align=center, yshift=-1cm] {Packet needed \\ $P$ = 0?};
	\node (ackrx) [io, left of=ack, xshift=-4cm, align=center] {Ack received \\ update $P$};

	\draw[arrow] (start) -- (setup);
	\draw[arrow] (setup) -- (filesetup);
	\draw[arrow] (filesetup) -- (while);
	\draw[arrow] (while) -- node[anchor=west] {yes} (read);
	\draw[arrow] (while) -- node[anchor=east] {no} (stop);
	\draw[arrow] (read) -- (send);
	\draw[arrow] (send) -- (waitack);
	\draw[arrow] (ackrx) -- (ack);
	\draw[arrow] (ack) -- node[anchor=east] {no} (send);
	\draw[arrow] (ack) -| node[anchor=west] {yes} ([xshift=+1cm]send.south east) |- node[anchor=west] {$b$++} (while);
\end{tikzpicture}
\caption{Sender flowchart}
\label{fig:sender}

\end{subfigure}%
%\hfill <-- it is superfluous 
\begin{subfigure}{0.5\hsize}\centering

	% picture of realization diagram
	\begin{tikzpicture}[node distance = 2cm,scale=0.8, every node/.style={scale=0.8}]
		\node (start) [startstop] {Init};
		\node (setup) [process, below of=start, yshift=-0.5cm, align=center] {Open socket, initialize \\ useful variables,\\\texttt{decoded} = \texttt{false},\\received packet $R=0$};
		\node (while) [decision, below right of=setup, yshift=-1.5cm, xshift=2cm,align=center] {File download\\completed?};
		\node (stop) [startstop, below left of=setup, yshift=-1.5cm, xshift=-2cm] {Stop};

		\node (waitpck) [process, below of=while, yshift=-1cm, align=center] {Wait for packet\\start timer T};
		
		\node (recv) [io, below of=stop, yshift=-1cm, align=center] {Receive packet\\$R= R+1$};

		\node (numpck) [decision, below of=recv, yshift=-0.3cm, align=center] {$R<N$?};

		\node (decode) [process, below of=numpck, align=center] {Decode};

		\node (success) [decision, below of=decode, align=center] {Successful\\decoding?};

		\node (write) [process, below of=success, yshift=-1cm, align=center] {Write decoded chunk\\Send ack with $P=0$};

		\node (sendack) [process, below of=waitpck, yshift=-0.3cm, align=center] {Send ack with\\$P=N-R$};

		\node (timer) [io, below of=sendack, align=center] {Timer expired};

		\node (sendun) [process, below of=timer, align=center] {Send ack with\\$P=N$};

		\draw[arrow] (start) -- (setup);
		\draw[arrow] (setup) -- (while);
		\draw[arrow] (while) -- node[anchor=west] {no, T=new\_block} (waitpck);
		\draw[arrow] (while) -- node[anchor=north] {yes} (stop);
		\draw[arrow] (recv) -- (numpck);
		\draw[arrow] (numpck) -- node[anchor=east] {yes} (decode);
		\draw[arrow] (numpck) -- node[anchor=south, yshift=+0.1cm] {no, T=pg} (waitpck);
		\draw[arrow] (decode) -- (success);
		\draw[arrow] (success) -- node[anchor=east] {yes} (write);
		\draw[arrow] (success) -- node[anchor=north] {no} (sendun);
		\draw[arrow] (sendun) -| node[anchor=south, yshift=-0.7cm] {T=ack} ([xshift=+1cm]timer.east) |-  (while);
		\draw[arrow] (timer) -- (sendack);
		\draw[arrow] (sendack) -- node[anchor=east] {T=ack} (waitpck);
		\draw[arrow] (write) -| ([xshift=+2cm]timer.east) |- (while.north);
		
		%\draw[arrow] (ackrx) -- (ack);
		%\draw[arrow] (ack) -- node[anchor=east] {no} (send);
		%\draw[arrow] (ack) -| node[anchor=west] {yes} ([xshift=+0.25cm]send.south east) |- node[anchor=west] {$b$++} (while);
	\end{tikzpicture}
	\caption{Receiver flowchart. T can be a timer of three different kind, pg for packet gap detection, ack for RTT, new\_block for new blocks (RTT + read + encoding)}
	\label{fig:receiver}

\end{subfigure}

\caption{Sender and receiver flowchart}
\label{fig:flowchart}
\end{sidewaysfigure}



\section{Implementation of RF and LT}
\subsection{RF}
Once the received has collected a suitable amount of packets, these packets are passed to the \texttt{decoding\_function}. This functions arranges the encoding vectors in a matrix and tries to invert it using Gau{\ss}-Jordan elimination. In order to get an efficient implementation of this algorithm in $\mathds{G}\mathds{F}(2)$ we used objects of class \texttt{mat\_GF2} which is implemented in library \texttt{NTL} ~\cite{NTL}. We then purge the all 0s rows, and check if on left part of the matrix we have a complete identity matrix of order K. If it's the case, then we extract the right part of the matrix  (i.e. we erase the identity matrix), which gives the inverse of the initial matrix. Using the inverse, we can then apply the inverse XOR transformation (decoding operation) on the encoded packet payloads, and obtain the original uncoded payloads. In the other case, the function returns the number of missing rows to get K independent ones. The inversion algorithm is then re-applied with the original plus the additional packets (unfortunately, there is no way to exploit the partial results of the failed Gau{\ss}-Jordan elimination to reduce the computations for the successive trials: the complexity is the same as starting from scratch).
\subsection{LT}
For the implementation of LT, we have to represent the graph of message passing. This is done by using two adjacency lists. This choice have been made in order to minimize the computing time of the algorithm, even though it results quite expensive in terms of memory usage. Packets are then resolved according to the message passing algorithm. In this case, if the algorithm fails, there is no easy way to determine how many independent encoding vectors would still be needed in oder to complete the decoding. This consideration and the fact that theoretically, for K sufficiently large, this event would be very unlikely, encouraged us to opt for a simpler system, in which if the decoding fails, then the receiver simply asks for another group of N packets to the sender. 
\section{Results and conclusions}
\lipsum[1]


\begin{thebibliography}{10}
\expandafter\ifx\csname url\endcsname\relax
  \def\url#1{\texttt{#1}}\fi
\expandafter\ifx\csname urlprefix\endcsname\relax\def\urlprefix{URL }\fi
\expandafter\ifx\csname href\endcsname\relax
  \def\href#1#2{#2} \def\path#1{#1}\fi
  
\bibitem{rfc6298} IETF, RFC 6298, Computing TCP's Retransmission Timer, June 2011
\bibitem{NTL} http://www.shoup.net/ntl/
\end{thebibliography}

\end{document}
