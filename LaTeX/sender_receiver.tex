\section{Technical Approach}\label{sec:impl}
\setlength{\abovecaptionskip}{10pt plus 3pt minus 2pt}

The objective of the project is to implement a network coding system that works and guarantees reliable transmission of files over different communication systems. With such a software, we will be able to assess the performance of network coding schemes in various setups, thus studying the trade-offs between different parameters (redundancy, retransmission timeouts) and metrics (goodput, efficiency).

The system is composed of two main C++ scripts, sender and receiver, and of some utility classes which handle coding/decoding, time interval estimation, packet abstraction. We implemented two different versions of the network coding system, one which relies on Random Fountain (RF), and another that uses Luby Transform coding (LT). We will show that the latter offers a better performance at a price of a more complex implementation. Let's describe as first the utilities, then how sender and receiver work (i.e. the retransmission mechanism) and finally the details on the implementation of the network coding algorithms.

\textbf{Utilities} -- The class \texttt{NCpacket} is an abstraction of an encoded packet, with a private variable (a \newline \texttt{NCpacketContainer} struct) that stores an header of 32 bit, a sequence number (\texttt{blockID}) of 8 bit, and the payload (i.e. the encoded data) as an array of fixed size. 
%A \texttt{NCpacket} can be created either from these three parameters or with the \texttt{blockID} and the whole chunk of uncoded data, and performs the encoding inside the constructor. 
The header represents a seed which is given to a pseudo random number generator (pRNG) in order to create the same encoding vector at sender and receiver side. 

The RF implementation directly creates encoded \texttt{NCpacket} objects in sender and receiver main methods.
The LT implementation, instead, uses a \textit{factory paradigm} to generate \texttt{NCpacket} objects, i.e. it does not directly call the object constructor but creates an helper, \texttt{NCpacketHelper}, which is initialized when the main method of sender (and receiver) is called. This allows to generate only once the Robust Soliton Distribution (RSD) needed to perform coding and the C++ objects of the \texttt{random} class. \texttt{NCpacketHelper} objects have a method that from the seed generates a vector (of variable size) with the position of ones in the encoding vector (which is much more efficient than handling the whole encoding vector, with few ones and thousands of zeros).

The \texttt{TimeCounter} class is used to perform estimation of time intervals, using the approach inspired to RFC 6298~\cite{rfc6298}. The time intervals of interest are the ones between the reception of two packets (in order to perform packet gap detection) and the RTT (to estimate whether an ACK was received by the sender). RFC 6298 proposes a method to estimate RTT for TCP connection, based on some filtering of raw measurements. However, the order of magnitude of the quantities of interest is much smaller than the minimum value that is returned by an estimator working with the rules described in~\cite{rfc6298}. Therefore some changes have to be made. Let $s_{est}$ be the smoothed estimate of the quantity of interest, $s_{var}$ an estimate of the variance, $s$ a new measurement. Before any measurement is taken, $s_{est}$ is initialized at 50 ms and $s_{var} = s_{est}/2$. Then, once a new value $s$ is available, these two variables are updated as follows
\begin{equation}
	s_{var} = (1-\beta) s_{var}  + \beta |s_{est} - s|
\end{equation}
\begin{equation}
	s_{est} = (1 - \alpha) s_{est} + \alpha s
\end{equation}
where $\alpha = 1/8$, $\beta = 1/4$ as suggested in~\cite{rfc6298}.

The value returned by the \texttt{TimeCounter} object is finally
\begin{equation}
	e = \max\left\{M, s_{est} + K\times s_{var}  \right\} \mbox{ ms}
\end{equation}
where a granularity of minimum $M$ ms is set (in~\cite{rfc6298} $M = 1000$ ms, we used $M = 1$ ms or $M=100$ ms) and $K = 4$. 

%\subsubsection*{Sender}
\textbf{Sender} -- Flow diagram for the sender is in Fig.~\ref{fig:sender}. The retransmission policy is a stop and wait (S\&W) per block, i.e. until a block is not successfully received new encoded packets for that block are sent. Both for the RF and LT implementations a rate $K/N$ is set to guarantee a certain probability of successful decoding. 

Given a certain $K$, the sender reads a portion of the file and creates $N$ encoded packets (as \texttt{NCpacket} objects). Then it sends them as UDP datagrams and starts waiting for an acknowledgment from the receiver. When an ACK is received, the sender checks for the number $P$ of packets still needed. If this is 0, it continues with the next block to be sent, otherwise it sends $P$ new encoded packets. In the payload of the first transmitted packet the sender reserves 4 bytes to communicate to the receiver the file size, so that this terminates correctly once the whole file is received. 

%\subsubsection*{Receiver}
\textbf{Receiver} -- The receiver flow diagram is in Fig.~\ref{fig:receiver}. 
For each block of data, the receiver waits for the reception of $N$ packets, then it attempts to decode them. If successful, it appends to the output file the decoded chunk of data and starts waiting for the next block. If not, it behaves differently in LT and RF implementations. In the latter, when running Gau{\ss} elimination, it is possible to know the number $P$ of linearly dependent encoding vectors. Therefore the receiver sends an ACK to the sender carrying $P$. When new packets are received, they are all used for decoding, until $K$ linearly independent packets are found. 
With LT, instead, this is not trivial and therefore we chose to query an entire new block by sending an ACK with $P=N$ to the sender. A possible improvement is to retransmit only a fraction of the $N$ packets and combine them with a random subset of the ones already received. However, if the ratio $K/N$ is low enough, the probability of unsuccessful decoding can be made as small as desired. 

The receiver uses different timers for different events. The first (called \texttt{pg} in Fig.~\ref{fig:receiver}) measures the time between the reception of two packets, in order to perform packet gap detection. If a packet is expected and the timer expires, the receiver assumes it was lost. Since the receiver has to collect $N$ packets before decoding, if the number of received packets $R$ is lower than $N$ and the timer expires, an ACK is transmitted to the sender specifying the number of needed packets $P = N - R$. The second timer (\texttt{ACK}) is used when waiting for the first packet after an ACK. It thus waits for the ACK to be received, new packets to be encoded and for the first one to be delivered at the receiver (this interval is defined as a \textit{round trip time} (RTT)). If it expires, the ACK is sent again. Both these timers use estimates provided by \texttt{TimeCounter} objects. For the second timer, however, a scaling factor is applied to the estimate $e$ to account for the fact that at each (re)transmission a different number of packets $P$ must be encoded, or a new chunk is read from file, thus different delays may be experienced. We observed that when the plain estimate $e$ was used the system was very aggressive and unneeded retransmissions were triggered. Therefore we applied the scaling factor of $100 \cdot P / N$ when $P < N$ or $10$ for $P = N$ which we observed offering a good tolerance against undesired retransmission while not degrading the goodput and throughput.

\textbf{PER estimation at sender} -- 
The sender has also a \texttt{PER\_MODE}. If this is enabled it estimates the \textit{Packet Error Rate} (PER) of the channel and tries to adapt to it. The estimation is done with a moving average on a window of past block transmissions. In particular the sender records the average number of packets $P_b$ needed to successfully transmit each block, and estimates the PER as $\widehat{PER} = 1 - N/P_b$. Then, once the sender needs to send $P$ packets, it sends $P/(1-\widehat{PER})$ packets trying to prevent channel losses without needing retransmissions. This lowers the efficiency of the system, but decreases the time needed to send the file. The estimation works better with the RF setup, since it transmits many more blocks than the LT one, of smaller size, thus it reacts more quickly to changes in the PER.

\begin{sidewaysfigure}
\captionsetup{justification=centering,margin=3cm}

\begin{subfigure}{0.5\hsize}\centering
    \begin{tikzpicture}[node distance = 2cm, scale=0.8, every node/.style={scale=0.8}]
	\node (start) [startstop] {Init};
	\node (setup) [process, below of=start, align=center] {Open socket, initialize\\useful variables, open input file};
	\node (filesetup) [process, below of=setup, align=center, yshift=-1cm] {Compute file size $F$\\ number of blocks $B=\ceil{F/K}$\\set number of tx block $b = 0$};
	\node (while) [decision, below of=filesetup, yshift=-1cm] {$b < B$?};
	\node (stop) [startstop, below left of=while, yshift=-1cm, xshift=-2cm] {Stop};
	\node (read) [process, below right of=while, yshift=-1cm, xshift=1.2cm, align=center] {Read data, set\\packets needed $P = N$};
	\node (send) [process, below of=read] {Encode and send $P$ packets};
	\node (waitack) [process, left of=send, xshift=-4cm] {Wait for ACK};
	\node (ACK) [decision, below of=send, align=center, yshift=-1cm] {Packet needed\\$P$ = 0?};
	\node (ackrx) [process, left of=ACK, xshift=-4cm, align=center] {Update $P$};

	\draw[arrow] (start) -- (setup);
	\draw[arrow] (setup) -- (filesetup);
	\draw[arrow] (filesetup) -- (while);
	\draw[arrow] (while) -- node[anchor=west] {yes} (read);
	\draw[arrow] (while) -- node[anchor=east] {no} (stop);
	\draw[arrow] (read) -- (send);
	\draw[arrow] (send) -- (waitack);
	\draw[arrow] (ackrx) -- (ACK);
	\draw[arrow] (ACK) -- node[anchor=east] {no} (send);
	\draw[arrow] (ACK) -| node[anchor=west] {yes} ([xshift=+1cm]send.south east) |- node[anchor=west] {$b$++} (while);
	\draw[darrow] (waitack) -- node[anchor=west] {Ack received} (ackrx);
\end{tikzpicture}
\caption{Sender flowchart}
\label{fig:sender}

\end{subfigure}%
%\hfill <-- it is superfluous 
\begin{subfigure}{0.5\hsize}\centering

	% picture of realization diagram
	\begin{tikzpicture}[node distance = 2cm,scale=0.8, every node/.style={scale=0.8}]
		\node (start) [startstop] {Init};
		\node (setup) [process, below of=start, yshift=-0.5cm, align=center] {Open socket, initialize\\useful variables,\\received packet $R=0$};
		\node (while) [decision, below right of=setup, yshift=-1.5cm, xshift=2cm,align=center] {File download\\completed?};
		\node (stop) [startstop, below left of=setup, yshift=-1.5cm, xshift=-2cm] {Stop};

		\node (waitpck) [process, below of=while, yshift=-1cm, align=center] {Wait for packet,\\start timer T};
		
		\node (recv) [process, below of=stop, yshift=-1cm, align=center] {$R= R+1$};

		\node (numpck) [decision, below of=recv, yshift=-0.3cm, align=center] {$R<N$?};

		\node (decode) [process, below of=numpck, align=center] {Decode};

		\node (success) [decision, below of=decode, align=center] {Successful\\decoding?};

		\node (write) [process, below of=success, yshift=-1cm, align=center] {Write decoded chunk,\\send ACK with $P=0$};

		\node (timer) [below of=waitpck, align=center] {};

		\node (sendack) [process, below of=timer, yshift=-0.3cm, align=center] {Send ACK with\\$P=N-R$};

		\node (sendun) [process, below of=sendack, align=center] {Send ACK with\\$P=N$};

		\draw[arrow] (start) -- (setup);	
		\draw[arrow] (setup) -- (while);
		\draw[arrow] (while) -- node[anchor=west] {no, T=\texttt{ACK}} (waitpck);
		\draw[arrow] (while) -- node[anchor=north] {yes} (stop);
		\draw[arrow] (recv) -- (numpck);
		\draw[arrow] (numpck) -- node[anchor=east] {no} (decode);
		\draw[arrow] (numpck) -- node[anchor=south, yshift=+0.1cm] {yes, T=\texttt{pg}} (waitpck);
		\draw[arrow] (decode) -- (success);
		\draw[arrow] (success) -- node[anchor=east] {yes} (write);
		\draw[arrow] (success) -- node[anchor=north] {no} (sendun);
		\draw[arrow] (sendun) -| node[anchor=south, yshift=-0.7cm] {T=\texttt{ACK}} ([xshift=+2cm]sendack.east) |-  (while);
		\draw[arrow] (sendack) -- node[anchor=east] {T=\texttt{ACK}} (waitpck);
		\draw[arrow] (write) -| ([xshift=+3cm]sendack.east) |- ([yshift=+1cm]while.north east) -| (while.north);

		\draw[darrow] (waitpck) -- node[anchor=south] {Receive packet} (recv);
		\draw[darrow] (waitpck) -| ([xshift=+1cm]waitpck.south east) |- node[anchor=north east, yshift=0.6cm, rotate=270] {Timer expired} (sendack);

		
		%\draw[arrow] (ackrx) -- (ACK);
		%\draw[arrow] (ACK) -- node[anchor=east] {no} (send);
		%\draw[arrow] (ACK) -| node[anchor=west] {yes} ([xshift=+0.25cm]send.south east) |- node[anchor=west] {$b$++} (while);
	\end{tikzpicture}
	\caption{Receiver flowchart. T can be a timer of two different kinds: \texttt{pg} for packet gap detection and \texttt{ACK} for RTT (network RTT + encoding of new blocks)}
	\label{fig:receiver}

\end{subfigure}

\caption{Sender and receiver flowcharts. The dashed arrows represent possible flows triggered by specific events.}
\label{fig:flowchart}
\end{sidewaysfigure}

\subsection{Network Coding implementation}
\textbf{Random Fountain} -- Once the receiver has collected a suitable amount of packets, these are passed to the \texttt{decoding\_function}. This function arranges the encoding vectors in a matrix and tries to invert it using Gau{\ss}-Jordan elimination. In order to get an efficient implementation of this algorithm in $\mathds{G}\mathds{F}(2)$ we used objects of class \texttt{mat\_GF2} which is implemented in library \texttt{NTL} ~\cite{NTL}. We then purge the all 0s rows, and check if on left part of the matrix we have a complete identity matrix of order $K$. If that is the case, then we extract the right part of the matrix  (i.e. we erase the identity matrix), which gives the inverse of the initial matrix. Using the inverse matrix, we can then apply the inverse XOR transformation (decoding operation) on the encoded packet payloads, and obtain the original uncoded information. In the other case, the function returns the number of missing rows to get $K$ independent rows. The inversion algorithm is then re-applied with the original plus the additional encoding vectors (unfortunately, there is no way to exploit the partial results of the failed Gau{\ss}-Jordan elimination to reduce the computations for the successive trials: the complexity is the same as starting from scratch).

\textbf{Luby Transform} -- For the implementation of LT, we have to represent the graph of message passing. This is done by using two adjacency lists. This choice was made in order to minimize the computing time of the algorithm, even though it results quite expensive in terms of memory usage. Packets are then resolved according to the message passing algorithm. In this case, if the algorithm fails, there is no easy way to determine how many independent encoding vectors would still be needed in order to complete the decoding. This consideration and the fact that theoretically, for $K$ sufficiently large, this event would be very unlikely, encouraged us to opt for a simpler system, in which if the decoding fails, then the receiver simply asks for another group of $N$ packets to the sender. The parameters of the Robust Soliton Distribution have been set to $c=0.03$ and $\delta=0.5$, as suggested in ~\cite{RSD}.

\subsection{Complications found}
In the implementation of LT decoder we faced a trade-off between memory consumption and computational complexity. We chose to optimize for the latter, this however constrained the possible $N$ and $K$ values to $K \le 5500$. Moreover, our choice of guaranteeing a low probability of decoding failure, prevented us from exploring the same intervals of relative redundancy for all values of $K$. Another issue was caused by the implementation of the \texttt{rand} function in C++, which is different in different versions of the Standard Library. Therefore the same seed generates different pseudo random sequences when sender and receiver use two different operative systems. We adopted the more modern C++11 \texttt{random} header which contains a Mersenne Twister which is platform independent, however the same problem arises with \texttt{random} \texttt{distribution} classes.


